\documentclass[draft]{article}
\usepackage{fullpage}
\usepackage{amsmath}
\usepackage{amssymb}

\include{macros}

% math fonts
\newcommand{\mbb}[1]{\mathbb{#1}}
\newcommand{\mbf}[1]{\mathbf{#1}}
\renewcommand{\mit}[1]{\mathit{#1}}
\newcommand{\mrm}[1]{\mathrm{#1}}
\newcommand{\mtt}[1]{\mathtt{#1}}
\newcommand{\mcal}[1]{\mathcal{#1}}
\newcommand{\msf}[1]{\mathsf{#1}}

% text fonts
\newcommand{\ttt}[1]{\texttt{#1}}

% formatting
\newenvironment{stackAux}[2]{%
\setlength{\arraycolsep}{0pt}
\begin{array}[#1]{#2}}{
\end{array}}
\newenvironment{stackCC}{
\begin{stackAux}{c}{c}}{\end{stackAux}}
\newenvironment{stackCL}{
\begin{stackAux}{c}{l}}{\end{stackAux}}
\newenvironment{stackTL}{
\begin{stackAux}{t}{l}}{\end{stackAux}}
\newenvironment{stackTR}{
\begin{stackAux}{t}{r}}{\end{stackAux}}
\newenvironment{stackBC}{
\begin{stackAux}{b}{c}}{\end{stackAux}}
\newenvironment{stackBL}{
\begin{stackAux}{b}{l}}{\end{stackAux}}

\newcommand{\stagger}[2]{%
\begin{array}{ccc}%
\multicolumn{2}{l}{#1}&\\%
&\multicolumn{2}{r}{#2}%
\end{array}}

\newcommand{\axiom}[1]{{\displaystyle\strut{#1}}}
\newcommand{\infrule}[2]{{\frac{\displaystyle\strut{#1}}{\displaystyle\strut{#2}}}} 
\newcommand{\judge}[2]{\infrule{#1}{#2}}

\begin{document}

\title{Formal Specification of the ML Basis system}
\author{Copyright \copyright\ 2004\\
        Henry Cejtin, Matthew Fluet, Suresh Jagannathan, and Stephen Weeks}
\date{\today}
\maketitle
%
This document formally specifies the ML Basis system (MLB) in {\mlton}
used to program in the large.  The system has been designed to be a
natural extension of Standard ML, and the specification is given in
the style of The Definition of Standard ML~\cite{MTHM97} (henceforeth,
the Definition).  This section adopts (often silently) abbreviations,
conventions, definitions, and notation from the Definition.
%
\section{Syntax of MLB}
For MLB there are further reserved words, identifier classes and
derived forms.  There are no further special constants; comments and
lexical analysis are as for the Core and Modules.  The derived forms
appear in Appendix~\ref{sec:mlb:DerivedForms}.
%
\subsection{Reserved Words}
The following are the additional reserved words used in MLB.
\begin{displaymath}
\mtt{bas} \quad\quad \mtt{basis}
\end{displaymath}
Note that many of the reserved words from the Core and Modules are not
used by the grammar of MLB.  However, as the grammar includes
identifiers from the grammars of the Core and Modules, it is useful to
consider the reserved words from the Core and Modules to be reserved
in MLB as well.
%
\subsection{Identifiers}
The additional identifier class for MLB are $\mrm{BasId}$ (basis
identifiers).  Basis identifiers must be alphanumeric, not starting
with a prime.  The class of each identifier occurence is determined by
the grammatical rules which follow.  Henceforth, therefore, we
consider all identifier classes to be disjoint.
%
\subsection{Infixed operators}
The grammar of MLB does not directly admit fixity directives.
However, the static and dynamic semantics for MLB will import source
files that must be parsed in the scope of fixity directives and that
may introduce additional fixity directives into scope.
Figure~\ref{fig:mlb:S:FixityEnv} formalizes the Definition's notion of
\emph{infix status} as a \emph{fixity environment}.
\begin{figure}[h]
\begin{displaymath}
\begin{array}{rcl}
 & & \mrm{InfixStatus} = \{\mtt{nonfix}\} \cup \bigcup_{d \in \{0,\ldots,9\}} \{\mtt{infix}~d, \mtt{infixr}~d\} \\
\mit{FE} & \in & \mrm{FixEnv} = \mrm{VId} \xrightarrow{\mrm{fin}} \mrm{InfixStatus} \end{array}
\end{displaymath}
\caption{Fixity Environment}\label{fig:mlb:S:FixityEnv}
\end{figure}
%
\subsection{Grammar for MLB}
The phrase classes for MLB are shown in
Figure~\ref{fig:mlb:S:PhraseClasses}.
\begin{figure}[h]
\begin{displaymath}
\begin{array}{ll}
\mrm{BasExp} & \mbox{basis expressions} \\
\mrm{BasDec} & \mbox{basis-level declaration} \\
\mrm{BasBind} & \mbox{basis bindings} \\
\mrm{BStrBind} & \mbox{(basis) structure bindings} \\
\mrm{BSigBind} & \mbox{(basis) signature bindings} \\
\mrm{BFunBind} & \mbox{(basis) functor bindings}
\end{array}
\end{displaymath}
\caption{MLB Phrase Classes}\label{fig:mlb:S:PhraseClasses}
\end{figure}
We use the variable $\mit{basexp}$ to range over $\mrm{BasExp}$, etc.
The conventions adopted in presenting the grammatical rules for MLB
are the same as for the Core and Modules.  The grammatical rules are
shown in Figure~\ref{fig:mlb:S:GrammaticalRules}.
\begin{figure}[h]
\begin{displaymath}
\begin{array}{rcll}
\mit{basexp} & ::= & 
\mtt{bas}~ \mit{basdec} ~\mtt{end} 
& \mbox{basic} \\&& 
\mit{basid} 
& \mbox{basis identifier} \\&&
\mtt{let}~ \mit{basdec} ~\mtt{in}~ \mit{basexp} ~\mtt{end} 
& \mbox{local declaration} \\

\mit{basdec} & ::= & 
\mtt{basis}~ \mit{basbind}
& \mbox{basis} \\&&
\mtt{local}~ \mit{basdec}_1 ~\mtt{in}~ \mit{basdec}_2 ~\mtt{end} 
& \mbox{local} \\&&
\mtt{open}~ \mit{basid}_1 \cdots \mit{basid}_n 
& \mbox{open ($n \geq 1$)} \\&&
\mtt{structure}~ \mit{bstrbind}
& \mbox{(basis) structure binding} \\&&
\mtt{signature}~ \mit{bsigbind}
& \mbox{(basis) signature binding} \\&&
\mtt{functor}~ \mit{bfunbind}
& \mbox{(basis) functor binding} \\&&
\quad
& \mbox{empty} \\&&
\mit{basdec}_1~\langle\mtt{;}\rangle~\mit{basdec}_2 
& \mbox{sequential} \\&&
\msf{path.mlb} &
\mbox{import ML basis} \\&&
\msf{path.sml} 
& \mbox{import source} \\&&
\msf{path.sig} 
& \mbox{import signature} \\&&
\msf{path.fun} 
& \mbox{import source (functor)} \\

\mit{basbind} & ::= &  
\mit{basid} ~\mtt{=}~ \mit{basexp} ~\langle\mtt{and}~ \mit{basbind}\rangle \\
\mit{bstrbind} & ::= &  
\mit{strid}_1 ~\mtt{=}~ \mit{strid}_2 ~\langle\mtt{and}~ \mit{bstrbind}\rangle \\
\mit{bsigbind} & ::= &  
\mit{sigid}_1 ~\mtt{=}~ \mit{sigid}_2 ~\langle\mtt{and}~ \mit{bsigbind}\rangle \\
\mit{bfunbind} & ::= &  
\mit{funid}_1 ~\mtt{=}~ \mit{funid}_2 ~\langle\mtt{and}~ \mit{bfunbind}\rangle
\end{array}
\end{displaymath}
\caption{Grammar: Basis Expressions, Declarations, and Bindings}\label{fig:mlb:S:GrammaticalRules}
\end{figure}
%
\subsection{Syntactic Restrictions}
\begin{itemize}
\item No binding $\mit{basbind}$ may bind the same identifier twice.
\item No binding $\mit{bstrbind}$, $\mit{bsigbind}$ or $\mit{bfunbind}$ may bind the same identifier twice.
\item MLB may not be cyclic; i.e., successively replacing
  $\msf{path.mlb}$ with its parsed $\mrm{BasDec}$ must terminate.
\end{itemize}
%
\subsection{Parsing}
The static and dynamic semantics for MLB will interpret
$\msf{path.sml}$ as a parsed $\mrm{TopDec}$ and
$\msf{path.mlb}$ as a parsed $\mrm{BasDec}$.  Parsing a $\mrm{TopDec}$
takes a fixity environment as input and returns a fixity environment
as output; the output fixity environment corresponds to fixity
directives introduced by and whose scope is not limited by the parsed
$\mrm{TopDec}$.

Paths and parsers are given in Figure~\ref{fig:mlb:S:PathsParser}.  A
(fixed) $\mrm{Parser}$ $\mcal{P}$ provides the interpretation of
$\msf{path.sml}$ and $\msf{path.mlb}$ imports.
\begin{figure}[h]
\begin{displaymath}
\begin{array}{c}
\begin{array}{rcl}
\msf{path.sml} & \in & \mrm{SourcePath} \\
\msf{path.mlb} & \in & \mrm{MLBasisPath} 
\end{array} \\
\begin{array}{rcl}
\mcal{P} & \in & \mrm{Parser} = 
((\mrm{FixEnv} \times \mrm{SourcePath})
 \xrightarrow{\mrm{fin}} (\mrm{FixEnv} \times \mrm{TopDec})) 
\times 
(\mrm{MLBasisPath} \xrightarrow{\mrm{fin}} \mrm{BasDec}) 
\end{array}
\end{array}
\end{displaymath}
\caption{Parser}\label{fig:mlb:S:PathsParser}
\end{figure}
%
For a file extension $\msf{.ext}$, $\msf{path.ext}$ denotes either an
absolute path or a relative path (relative to the $\mrm{BasDec}$ being
parsed) to a file in the underlying file system.  Paths that denote the same
file in the underlying file system are considered equal, though they may
have distinct textual representations.  An implementation
may allow additional extensions (e.g., $\mtt{.ML}$, $\mtt{.fun}$,
$\mtt{.sig}$) in elements of $\mrm{SourcePath}$.  An implementation
may additionally allow path variables to appear in
paths.  $\mrm{Parser}$ could be refined by a \emph{current working
directory}, to formally specify the interpretation of relative paths,
and an \emph{path map}, to formally specify the
interpretation of path variables, but the above suffices
for the development in the following sections.
%
\section{Static Semantics for MLB}
%
\subsection{Semantic Objects}
The simple objects for the MLB static semantics are exactly as for
Modules.  The compound objects are those for Modules, augmented by
those in Figure~\ref{fig:mlb:SS:CompoundObjects}.
\begin{figure}[h]
\begin{displaymath}
\begin{array}{rcl}
\mit{BE} & \in & \mrm{BasEnv} = \mrm{BasId} \xrightarrow{\mrm{fin}} \mrm{MBasis} \\
\mit{M} ~\mrm{or}~ \mit{FE},\mit{BE},\mit{B} & \in & 
\mrm{MBasis} = \mrm{FixEnv} \times \mrm{BasEnv} \times \mrm{Basis} \\
\Psi & \in & \mrm{BasCache} = \mrm{MLBasisPath} \xrightarrow{\mrm{fin}} \mrm{MBasis} 
\end{array}
\end{displaymath}
\caption{Compound Semantic Objects}\label{fig:mlb:SS:CompoundObjects}
\end{figure}
The operations of projection, injection and modification are as for
Modules.
%
\subsection{Inference Rules}
As for the Core and for Modules, the rules for MLB static semantics
allow sentences of the form
\begin{displaymath}
A \vdash \mit{phrase} \longrightarrow A'
\end{displaymath}
to be inferred.  Some hypotheses in rules are not of this form; they
are called \emph{side-conditions}.  The convention for options is as
in the Core and Modules semantics.

\vspace{2\parsep}
{\large\noindent
\textbf{Basis Expressions} \hfill
\fbox{$\mit{M}, \Psi \vdash \mit{basexp} \longrightarrow \mit{M}', \Psi'$}
}\nopagebreak

\begin{equation}
\judge{
\mit{M}, \Psi \vdash \mit{basdec} \longrightarrow \mit{M}', \Psi'
}{
\mit{M}, \Psi \vdash \mtt{bas}~ \mit{basdec} ~\mtt{end} \longrightarrow \mit{M}', \Psi'
}
\end{equation}

\begin{equation}
\judge{
\mit{M}(\mit{basid}) = \mit{M}'
}{
\mit{M}, \Psi \vdash \mit{basid} \longrightarrow \mit{M}', \Psi
}
\end{equation}

\begin{equation}
\label{eqn:mlb:SS:localDeclaration}
\judge{
\mit{M}, \Psi \vdash \mit{basdec} \longrightarrow \mit{M}_1, \Psi_1 \quad
\mit{M} \oplus \mit{M}_1, \Psi_1 \vdash \mit{basexp} \longrightarrow \mit{M}_2, \Psi_2
}{
\mit{M}, \Psi \vdash \mtt{let}~ \mit{basdec} ~\mtt{in}~ \mit{basexp} ~\mtt{end} \longrightarrow \mit{M}_2, \Psi_2
}
\end{equation}

\begin{samepage}
\noindent
\textit{Comments}:
\begin{itemize}
\item[(\ref{eqn:mlb:SS:localDeclaration})] The use of $\oplus$, here
  and elsewhere, ensures that the type names generated by the first
  sub-phrase are distinct from the names generated by the second sub-phrase.
\end{itemize}
\end{samepage}

\vspace{2\parsep}
{\large\noindent
\textbf{Basis-level Declarations} \hfill 
\fbox{$\mit{M}, \Psi \vdash \mit{basdec} \longrightarrow \mit{M}', \Psi'$}
}\nopagebreak

\begin{equation}
\judge{
\mit{M}, \Psi  \vdash \mit{basbind} \longrightarrow \mit{BE}', \Psi'
}{
\mit{M}, \Psi  \vdash \msf{basis}~ \mit{basbind} \longrightarrow \mit{BE}' ~\mrm{in}~ \mrm{MBasis}, \Psi'
}
\end{equation}

\begin{equation}
\judge{
\mit{M}, \Psi  \vdash \mit{basdec}_1 \longrightarrow \mit{M}_1, \Psi_1 \quad
\mit{M} \oplus \mit{M}_1, \Psi_1  \vdash \mit{basdec}_2 \longrightarrow \mit{M}_2, \Psi_2 \quad
}{
\mit{M}, \Psi  \vdash \mtt{local}~ \mit{basdec}_1 ~\mtt{in}~ \mit{basdec}_2 ~\mtt{end} \longrightarrow \mit{M}_2, \Psi_2
}
\end{equation}

\begin{equation}
\judge{
\mit{M}(\mit{basid}_1) = \mit{M}_1 \quad \cdots \quad
\mit{M}(\mit{basid}_n) = \mit{M}_n 
}{
\mit{M}, \Psi  \vdash \mtt{open}~ \mit{basid}_1 \cdots \mit{basid}_n \longrightarrow \mit{M}_1 \oplus \cdots \oplus \mit{M}_n, \Psi
}
\end{equation}

\begin{equation}
\judge{
\mit{B}~\mrm{of}~\mit{M} \vdash \mit{bstrbind} \longrightarrow SE
}{
\mit{M}, \Psi  \vdash \mtt{structure}~ \mit{bstrbind}
\longrightarrow \mit{SE} ~\mrm{in}~ \mrm{MBasis}, \Psi
}
\end{equation}

\begin{equation}
\judge{
\mit{B}~\mrm{of}~\mit{M} \vdash \mit{bsigbind} \longrightarrow G
}{
\mit{M}, \Psi  \vdash \mtt{signature}~ \mit{bsigbind}
\longrightarrow \mit{G } ~\mrm{in}~ \mrm{MBasis}, \Psi
}
\end{equation}

\begin{equation}
\judge{
\mit{B}~\mrm{of}~\mit{M} \vdash \mit{bfunbind} \longrightarrow F
}{
\mit{M}, \Psi  \vdash \mtt{functor}~ \mit{bfunbind}
\longrightarrow \mit{F} ~\mrm{in}~ \mrm{MBasis}, \Psi
}
\end{equation}

\begin{equation}
\judge{
}{
\mit{M}, \Psi  \vdash \quad \longrightarrow \{\} ~\mrm{in}~ \mrm{MBasis}, \Psi
}
\end{equation}

\begin{equation}
\judge{
\mit{M}, \Psi  \vdash \mit{basdec}_1 \longrightarrow \mit{M}_1, \Psi_1 \quad
\mit{M} \oplus \mit{M}_1, \Psi_1  \vdash \mit{basdec}_2 \longrightarrow \mit{M}_2, \Psi_2 
}{
\mit{M}, \Psi  \vdash \mit{basdec}_1 ~\langle\mtt{;}\rangle~ \mit{basdec}_2 \longrightarrow \mit{M}_1 \oplus \mit{M}_2, \Psi_2
}
\end{equation}

\begin{equation}
\label{eqn:mlb:SS:path.sml}
\judge{
\mcal{P}(\mit{FE}~\mrm{of}~\mit{M}, \msf{path.sml}) = (\mit{FE}', \mit{topdec}) \quad
\mit{B}~\mrm{of}~\mit{M} \vdash \mit{topdec} \Rightarrow \mit{B}'
}{
\mit{M}, \Psi  \vdash \msf{path.sml}  \longrightarrow (\mit{FE}',\{\},\mit{B}'), \Psi
}
\end{equation}

\begin{equation}
\judge{
\Psi(\msf{path.mlb}) = \mit{M}'
}{
\mit{M}, \Psi  \vdash \msf{path.mlb}  \longrightarrow \mit{M}', \Psi
}
\end{equation}

\begin{equation}
\judge{
\msf{path.mlb} \notin \mrm{Dom}~\Psi \quad
\mcal{P}(\msf{path.mlb}) = \mit{basdec} \quad
\{\} ~\mrm{in}~ \mrm{MBasis}, \Psi  \vdash \mit{basdec} \longrightarrow \mit{M}', \Psi'
}{
\mit{M}, \Psi  \vdash \msf{path.mlb}  \longrightarrow \mit{M}', \Psi' + \{\msf{path.mlb} \mapsto \mit{M}'\} 
}
\end{equation}

\begin{samepage}
\noindent
\textit{Comments}:
\begin{itemize}
\item[(\ref{eqn:mlb:SS:path.sml})] 
Note the use of the Definition's 
$\mit{B} \vdash \mit{topdec} \Rightarrow \mit{B}'$.
\end{itemize}
\end{samepage}

\vspace{2\parsep}
{\large\noindent
\textbf{Basis Bindings} \hfill 
\fbox{$\mit{M}, \Psi \vdash \mit{basbind} \longrightarrow \mit{BE}', \Psi'$}
}\nopagebreak

\begin{equation}
\judge{
\mit{M}, \Psi \vdash \mit{basexp} \longrightarrow \mit{M}', \Psi' \quad
\langle\mit{M} + \mrm{tynames}~\mit{M}', \Psi' \vdash \mit{basbind} \longrightarrow \mit{BE}'', \Psi''\rangle
}{
\mit{M}, \Psi  \vdash \mit{basid} ~\mtt{=}~ \mit{basexp} ~\langle\mtt{and}~\mit{basbind}\rangle \longrightarrow 
\{\mit{basid} \mapsto \mit{M}'\} \langle+ \mit{BE}''\rangle, \Psi'\langle'\rangle
}
\end{equation}

\vspace{2\parsep}
{\large\noindent
\textbf{(Basis) Structure Bindings} \hfill 
\fbox{$\mit{B} \vdash \mit{bstrbind} \longrightarrow \mit{SE}$}
}\nopagebreak

\begin{equation}
\label{eqn:mlb:SS:bstrbind}
\judge{
\mit{B}(\mit{strid}_2) = E \quad
\langle\mit{B} + \mrm{tynames}~\mit{E} \vdash \mit{bstrbind} \longrightarrow \mit{SE}\rangle
}{
\mit{B} \vdash \mit{strid}_1 ~\mtt{=}~ \mit{strid}_2 ~\langle\mtt{and}~\mit{bstrbind}\rangle \longrightarrow 
\{\mit{strid}_1 \mapsto \mit{E}\} \langle+ \mit{SE}\rangle
}
\end{equation}

\begin{samepage}
\noindent
\textit{Comments}:
\begin{itemize}
\item[(\ref{eqn:mlb:SS:bstrbind})] Note that $\mit{bstrbind} \subset
\mit{strbind}$.  Hence, this rule can be derived from the
Definition's $\mit{B} \vdash \mit{strbind} \Rightarrow \mit{SE}$.
\end{itemize}
\end{samepage}

\vspace{2\parsep}
{\large\noindent
\textbf{(Basis) Signature Bindings} \hfill 
\fbox{$\mit{B} \vdash \mit{bsigbind} \longrightarrow \mit{G}$}
}\nopagebreak

\begin{equation}
\label{eqn:mlb:SS:bsigbind}
\judge{
\begin{stackCC}
\mit{B}(\mit{sigid}_2) = \Sigma \quad \Sigma = (\mit{T})\mit{E} \quad
\mit{T} \cap (\mit{T}~\mrm{of}~\mit{B}) = \emptyset \\
\mit{T} = \mrm{tynames}~\mit{E} \setminus (\mit{T}~\mrm{of}~\mit{B}) \quad
\langle\mit{B} \vdash \mit{bsigbind} \longrightarrow \mit{G}\rangle
\end{stackCC}
}{
\mit{B} \vdash \mit{sigid}_1 ~\mtt{=}~ \mit{sigid}_2 ~\langle\mtt{and}~\mit{bsigbind}\rangle \longrightarrow 
\{\mit{sigid}_1 \mapsto \Sigma\} \langle+ \mit{G}\rangle
}
\end{equation}

\begin{samepage}
\noindent
\textit{Comments}:
\begin{itemize}
\item[(\ref{eqn:mlb:SS:bsigbind})] Note that $\mit{bsigbind} \subset
\mit{sigbind}$.  Hence, this rule can be derived from the
Definition's $\mit{B} \vdash \mit{sigbind} \Rightarrow \mit{G}$.  As such, the
following comment from the Definition applies:
\begin{quote}
The bound names of $\mit{B}(\mit{sigid}_2)$ can always be renamed to
satisfy $\mit{T} \cap (\mit{T}~\mrm{of}~\mit{B}) = \emptyset$, if necessary.
\end{quote}
\end{itemize}
\end{samepage}

\vspace{2\parsep}
{\large\noindent
\textbf{(Basis) Functor Bindings} \hfill 
\fbox{$\mit{B} \vdash \mit{bfunbind} \longrightarrow \mit{F}$}
}\nopagebreak

\begin{equation}
\judge{
\begin{stackCC}
\mit{B}(\mit{funid}_2) = \Phi \quad \Phi = (\mit{T})(\mit{E},(\mit{T}')\mit{E}') \quad
\mit{T} \cap (\mit{T}~\mrm{of}~\mit{B}) = \emptyset \\
\mit{T}' = \mrm{tynames}~\mit{E}' \setminus ((\mit{T}~\mrm{of}~\mit{B}) \cup \mit{T}) \quad
\langle\mit{B} \vdash \mit{bfunbind} \longrightarrow \mit{F}\rangle
\end{stackCC}
}{
\mit{B} \vdash \mit{funid}_1 ~\mtt{=}~ \mit{funid}_2 ~\langle\mtt{and}~\mit{bfunbind}\rangle \longrightarrow 
\{\mit{funid}_1 \mapsto \Phi\} \langle+ \mit{F}\rangle
}
\end{equation}
%
\section{Dynamic Semantics for MLB}
%
\subsection{Reduced Syntax}
The syntax of MLB is unchanged for the purposes of the dynamic
semantics for MLB.  However, the $\mrm{Parser}$ $\mcal{P}$ returns a
$\mit{topdec}$ in the reduced syntax of Modules.
%
\subsection{Compound Objects}
The compound objects for the MLB dynamic semantics, extra to those
for the Modules dynamic semantics, are shown in Figure~\ref{fig:mlb:DS:CompoundObjects}.
\begin{figure}[h]
\begin{displaymath}
\begin{array}{rcl}
\mit{BE} & \in & \mrm{BasEnv} = \mrm{BasId} \xrightarrow{\mrm{fin}} \mrm{MBasis} \\
\mit{M} ~\mrm{or}~ \mit{FE},\mit{BE},\mit{B} & \in & \mrm{MBasis} =
\mrm{FixEnv} \times \mrm{BasEnv} \times \mrm{Basis} \\
\Psi & \in & \mrm{BasCache} = \mrm{MLBasisPath} \xrightarrow{\mrm{fin}} \mrm{MBasis} 
\end{array}
\end{displaymath}
\caption{Compound Semantic Objects}\label{fig:mlb:DS:CompoundObjects}
\end{figure}
%
\subsection{Inference Rules}
The semantic rules allow sentences of the form
\begin{displaymath}
s, A \vdash \mit{phrase} \longrightarrow A', s'
\end{displaymath}
to be inferred, where $s$, $s'$ are the states before and after the
evaluation represented by the sentence.  Some hypotheses in rules are
not of this form; they are called \emph{side-conditions}. The
convention for options is as in the Core and Modules semantics.

The state and exception conventions are adopted as in the Core and
Modules dynamic semantics.  However, it can be shown that the only
MLB phrases whose evaluation may cause a side-effect or generate an
exception packet are of the form $\mit{basexp}$, $\mit{basdec}$ or
$\mit{basbind}$.

\vspace{2\parsep}
{\large\noindent
\textbf{Basis Expressions} \hfill 
\fbox{$\mit{M}, \Psi \vdash \mit{basexp} \longrightarrow \mit{M}', \Psi' / p$}
}\nopagebreak

\begin{equation}
\judge{
\mit{M}, \Psi \vdash \mit{basdec} \longrightarrow \mit{M}', \Psi'
}{
\mit{M}, \Psi \vdash \mtt{bas}~ \mit{basdec} ~\mtt{end} \longrightarrow \mit{M}', \Psi'
}
\end{equation}

\begin{equation}
\judge{
\mit{M}(\mit{basid}) = \mit{M}'
}{
\mit{M}, \Psi \vdash \mit{basid} \longrightarrow \mit{M}', \Psi
}
\end{equation}

\begin{equation}
\judge{
\mit{M}, \Psi \vdash \mit{basdec} \longrightarrow \mit{M}_1, \Psi_1 \quad
\mit{M} \oplus \mit{M}_1, \Psi_1 \vdash \mit{basexp} \longrightarrow \mit{M}_2, \Psi_2
}{
\mit{M}, \Psi \vdash \mtt{let}~ \mit{basdec} ~\mtt{in}~ \mit{basexp} ~\mtt{end} \longrightarrow \mit{M}_2, \Psi_2
}
\end{equation}

\vspace{2\parsep}
{\large\noindent
\textbf{Basis-level Declarations} \hfill 
\fbox{$\mit{M}, \Psi \vdash \mit{basdec} \longrightarrow \mit{M}', \Psi' / p$}
}\nopagebreak

\begin{equation}
\judge{
\mit{M}, \Psi  \vdash \mit{basbind} \longrightarrow \mit{BE}', \Psi'
}{
\mit{M}, \Psi  \vdash \msf{basis}~ \mit{basbind} \longrightarrow \mit{BE}' ~\mrm{in}~ \mrm{MBasis}, \Psi'
}
\end{equation}

\begin{equation}
\judge{
\mit{M}, \Psi  \vdash \mit{basdec}_1 \longrightarrow \mit{M}_1, \Psi_1 \quad
\mit{M} + \mit{M}_1, \Psi_1  \vdash \mit{basdec}_2 \longrightarrow \mit{M}_2, \Psi_2 \quad
}{
\mit{M}, \Psi  \vdash \mtt{local}~ \mit{basdec}_1 ~\mtt{in}~ \mit{basdec}_2 ~\mtt{end} \longrightarrow \mit{M}_2, \Psi_2
}
\end{equation}

\begin{equation}
\judge{
\mit{M}(\mit{basid}_1) = \mit{M}_1 \quad \cdots \quad
\mit{M}(\mit{basid}_n) = \mit{M}_n 
}{
\mit{M}, \Psi  \vdash \mtt{open}~ \mit{basid}_1 \cdots \mit{basid}_n \longrightarrow \mit{M}_1 + \cdots + \mit{M}_n, \Psi
}
\end{equation}

\begin{equation}
\judge{
\mit{B}~\mrm{of}~\mit{M} \vdash \mit{bstrbind} \longrightarrow SE
}{
\mit{M}, \Psi  \vdash \mtt{structure}~ \mit{bstrbind}
\longrightarrow \mit{SE} ~\mrm{in}~ \mrm{MBasis}, \Psi
}
\end{equation}

\begin{equation}
\judge{
\mrm{Inter}~(\mit{B}~\mrm{of}~\mit{M}) \vdash \mit{bsigbind} \longrightarrow G
}{
\mit{M}, \Psi  \vdash \mtt{signature}~ \mit{bsigbind}
\longrightarrow \mit{G } ~\mrm{in}~ \mrm{MBasis}, \Psi
}
\end{equation}

\begin{equation}
\judge{
\mit{B}~\mrm{of}~\mit{M} \vdash \mit{bfunbind} \longrightarrow F
}{
\mit{M}, \Psi  \vdash \mtt{functor}~ \mit{bfunbind}
\longrightarrow \mit{F} ~\mrm{in}~ \mrm{MBasis}, \Psi
}
\end{equation}

\begin{equation}
\judge{
}{
\mit{M}, \Psi  \vdash \quad \longrightarrow \{\} ~\mrm{in}~ \mrm{MBasis}, \Psi
}
\end{equation}

\begin{equation}
\judge{
\mit{M}, \Psi  \vdash \mit{basdec}_1 \longrightarrow \mit{M}_1, \Psi_1 \quad
\mit{M} + \mit{M}_1, \Psi_1  \vdash \mit{basdec}_2 \longrightarrow \mit{M}_2, \Psi_2 
}{
\mit{M}, \Psi  \vdash \mit{basdec}_1 ~\langle\mtt{;}\rangle~ \mit{basdec}_2 \longrightarrow \mit{M}_1 \oplus \mit{M}_2, \Psi_2
}
\end{equation}

\begin{equation}
\label{eqn:mlb:DS:path.sml}
\judge{
\mcal{P}(\mit{FE}~\mrm{of}~\mit{M}, \msf{path.sml}) = (\mit{FE}', \mit{topdec}) \quad
\mit{B}~\mrm{of}~\mit{M} \vdash \mit{topdec} \Rightarrow \mit{B}'
}{
\mit{M}, \Psi  \vdash \msf{path.sml}  \longrightarrow (\mit{FE}',\{\},\mit{B}'), \Psi
}
\end{equation}

\begin{equation}
\judge{
\Psi(\msf{path.mlb}) = \mit{M}'
}{
\mit{M}, \Psi  \vdash \msf{path.mlb}  \longrightarrow \mit{M}', \Psi
}
\end{equation}

\begin{equation}
\judge{
\msf{path.mlb} \notin \mrm{Dom}~\Psi \quad
\mcal{P}(\msf{path.mlb}) = \mit{basdec} \quad
\{\} ~\mrm{in}~ \mrm{MBasis}, \Psi  \vdash \mit{basdec} \longrightarrow \mit{M}', \Psi'
}{
\mit{M}, \Psi  \vdash \msf{path.mlb}  \longrightarrow \mit{M}', \Psi' + \{\msf{path.mlb} \mapsto \mit{M}'\} 
}
\end{equation}

\begin{samepage}
\noindent
\textit{Comments}:
\begin{itemize}
\item[(\ref{eqn:mlb:DS:path.sml})] 
Note the use of the Definition's 
$\mit{B} \vdash \mit{topdec} \Rightarrow \mit{B}'$.
\end{itemize}
\end{samepage}

\vspace{2\parsep}
{\large\noindent
\textbf{Basis Bindings} \hfill 
\fbox{$\mit{M}, \Psi \vdash \mit{basbind} \longrightarrow \mit{BE}', \Psi' / p$}
}\nopagebreak

\begin{equation}
\judge{
\mit{M}, \Psi \vdash \mit{basexp} \longrightarrow \mit{M}', \Psi' \quad
\langle\mit{M}, \Psi' \vdash \mit{basbind} \longrightarrow \mit{BE}'', \Psi''\rangle
}{
\mit{M}, \Psi  \vdash \mit{basid} ~\mtt{=}~ \mit{basexp} ~\langle\mtt{and}~\mit{basbind}\rangle \longrightarrow 
\{\mit{basid} \mapsto \mit{M}'\} \langle+ \mit{BE}''\rangle, \Psi'\langle'\rangle
}
\end{equation}

\vspace{2\parsep}
{\large\noindent
\textbf{(Basis) Structure Bindings} \hfill 
\fbox{$\mit{B} \vdash \mit{bstrbind} \longrightarrow \mit{SE}$}
}\nopagebreak

\begin{equation}
\label{eqn:mlb:DS:bstrbind}
\judge{
\mit{B}(\mit{strid}_2) = E \quad
\langle\mit{B} \vdash \mit{bstrbind} \longrightarrow \mit{SE}\rangle
}{
\mit{B} \vdash \mit{strid}_1 ~\mtt{=}~ \mit{strid}_2 ~\langle\mtt{and}~\mit{bstrbind}\rangle \longrightarrow 
\{\mit{strid}_1 \mapsto \mit{E}\} \langle+ \mit{SE}\rangle
}
\end{equation}

\begin{samepage}
\noindent
\textit{Comments}:
\begin{itemize}
\item[(\ref{eqn:mlb:DS:bstrbind})] Note that $\mit{bstrbind} \subset
\mit{strbind}$.  Hence, this rule can be derived from the
Definition's $\mit{B} \vdash \mit{strbind} \Rightarrow \mit{SE} / p$, noting that
the derivation may neither cause a side-effect nor generate an
exception packet.
\end{itemize}
\end{samepage}

\vspace{2\parsep}
{\large\noindent
\textbf{(Basis) Signature Bindings} \hfill 
\fbox{$\mit{IB} \vdash \mit{bsigbind} \longrightarrow \mit{G}$}
}\nopagebreak

\begin{equation}
\label{eqn:mlb:DS:bsigbind}
\judge{
\mit{IB}(\mit{sigid}_2) = I \quad 
\langle\mit{IB} \vdash \mit{bsigbind} \longrightarrow \mit{G}\rangle
}{
\mit{IB} \vdash \mit{sigid}_1 ~\mtt{=}~ \mit{sigid}_2 ~\langle\mtt{and}~\mit{bsigbind}\rangle \longrightarrow 
\{\mit{sigid}_1 \mapsto I\} \langle+ \mit{G}\rangle
}
\end{equation}

\begin{samepage}
\noindent
\textit{Comments}:
\begin{itemize}
\item[(\ref{eqn:mlb:DS:bsigbind})] Note that $\mit{bsigbind} \subset
\mit{sigbind}$.  Hence, this rule can be derived from the
Definition's $\mit{IB} \vdash \mit{sigbind} \Rightarrow \mit{G}$, noting that
the derivation may neither cause a side-effect nor generate an
exception packet.
\end{itemize}
\end{samepage}

\vspace{2\parsep}
{\large\noindent
\textbf{(Basis) Functor Bindings} \hfill 
\fbox{$\mit{B} \vdash \mit{bfunbind} \longrightarrow \mit{F}$}
}\nopagebreak

\begin{equation}
\judge{
\mit{B}(\mit{funid}_2) = (\mit{strid}:\mit{I},\mit{strexp},\mit{B}) \quad
\langle\mit{B} \vdash \mit{bfunbind} \longrightarrow \mit{F}\rangle
}{
\mit{B} \vdash \mit{funid}_1 ~\mtt{=}~ \mit{funid}_2 ~\langle\mtt{and}~\mit{bfunbind}\rangle \longrightarrow 
\{\mit{funid}_1 \mapsto (\mit{strid}:\mit{I},\mit{strexp},\mit{B})\} \langle+ \mit{F}\rangle
}
\end{equation}

\appendix
\section{Derived Forms}
\label{sec:mlb:DerivedForms}
Figure~\ref{fig:mlb:DF:bindings} shows derived forms for structure,
signature, and functor bindings in MLB.  These derived forms are
a useful shorthand for specifying import and export filters.

\begin{figure}[h]
\begin{center}
\begin{tabular}{|l|l|}
\multicolumn{1}{c}{Derived Form} &
\multicolumn{1}{c}{Equivalent Form} \\
\multicolumn{2}{c}{} \\
\multicolumn{2}{l}{\textbf{(Basis) Structure Bindings} $\mit{bstrbind}$} \\
\hline
$\mit{strid} ~\langle\mtt{and}~ \mit{bstrbind}\rangle$ &
$\mit{strid} ~\mtt{=}~ \mit{strid} ~\langle\mtt{and}~ \mit{bstrbind}\rangle$ \\
\hline
\multicolumn{2}{c}{} \\
\multicolumn{2}{l}{\textbf{(Basis) Signature Bindings} $\mit{bsigbind}$} \\
\hline
$\mit{sigid} ~\langle\mtt{and}~ \mit{bsigbind}\rangle$ &
$\mit{sigid} ~\mtt{=}~ \mit{sigid} ~\langle\mtt{and}~ \mit{bsigbind}\rangle$ \\
\hline
\multicolumn{2}{c}{} \\
\multicolumn{2}{l}{\textbf{(Basis) Functor Bindings} $\mit{bfunbind}$} \\
\hline
$\mit{funid} ~\langle\mtt{and}~ \mit{bfunbind}\rangle$ &
$\mit{funid} ~\mtt{=}~ \mit{funid} ~\langle\mtt{and}~ \mit{bfunbind}\rangle$ \\
\hline
\end{tabular}
\end{center}
\caption{Derived forms of (Basis) Structure, Signature, and Functor Bindings}\label{fig:mlb:DF:bindings}
\end{figure}

\bibliographystyle{alpha}
\bibliography{bib}

\end{document}
