\sec{Running on Cygwin/Windows}{cygwin}

{\mlton} uses the \htmladdnormallink{Cygwin}{http://www.cygwin.com/} emulation
layer to provide a Posix-like environment while running on a Windows machine.
To run {\mlton}, you must first install Cygwin on your machine.  To do this,
visit the Cygwin site from your Windows machine and run their {\tt setup.exe}
script.  Due to some recent bugfixes in the Cygwin dll, you must use version
1.3.11 or later.  As of this writing, that version does not exist, so you must
use a Cygwin developer snapshot from 2002-Mar-16 or later.

{\mlton} under Cygwin mostly behaves like {\mlton} under Linux.  There are,
however, a few missing features and known problems.

\begin{enumerate}

\item Profiling is disabled.

\item Due to several bugs in Cygwin's emulation of {\tt fork}, {\tt
Posix.Process.fork} is disabled.  Any use of {\tt fork} will raise
{\tt OS.SysErr}.  For idiomatic uses of {\tt fork} plus {\tt exec},
you can instead use the {\tt MLton.Process.spawn} family of functions, which
work on both Cygwin and Linux.

\item There are problems with {\tt mmap}ing large amounts of memory that may
cause programs with large heaps to fail.  Hopefully, the runtime will produce a
valid error message (``out of memor'') in this case, but, unfortunately, I have
seen segfaults as well.

\item I have seen some strangeness in Cygwin's emulation of signals and
signal handlers, but have not been able to pin it down.

\end{enumerate}

If you notice any of these problems, or others, and especially if you can
provide help, please report them to {\mltonmail}.

\subsec{Cross compiling applications from Linux to Cygwin/Windows}
       {cross-compiling}

With {\mlton} running on Linux, you can use the {\tt -host} flag to cross
compile applications and produce executables that run on Cygwin/Windows.
In order to use {\mlton} as a
cross compiler, you need to do several things.

\begin{enumerate}

\item Install the Cygwin dll in the Windows machine.

\item Install the GCC cross-compiler tools on your Linux machine.

\item Cross compile the {\mlton} runtime system for your Windows machine.

\end{enumerate}

To build a GCC cross-compiler toolset on your machine, you can use the script
{\tt bin/build-cross-gcc} available in the {\mlton} sources.  There are some
comments at the top of the script that tell you what to download and what
variables to set in order to build the toolset.  In particular, the {\tt target}
variable is important, since that is what you will pass to {\mlton}'s {\tt
-host} flag.

Once you have the toolset built, you should be able to test it by cross
compiling a simple hello world program on your Linux machine.
\begin{verbatim}
gcc -b i386-pc-cygwin -o hello-world.exe hello-world.c
\end{verbatim}
You should now be able to run {\tt hello-world.exe} from a Cygwin shell on your
Windows Machine.

Next, you must cross compile the {\mlton} runtime system and inform {\mlton} of
the availability of the new target.  The script {\tt bin/add-cross} from
the {\mlton} sources that will help you do this.  Please read the comments at
the top of the script.  Here is a sample run.
\begin{verbatim}
% add-cross
Making runtime.
Building print-constants executable.
You must now run print-constants.exe on the i386-pc-cygwin machine
and put the output in /home/sweeks/mlton/src/build/lib/i386-pc-cygwin/constants.
\end{verbatim}
Running {\tt add-cross} installs the cross-compiled runtime and creates a
cross-compiled executable, {\tt print-constants.exe}, which prints out all of
the constants that {\mlton} needs in order to implement the basis library.  The
final step is to run {\tt print-constants.exe} on your Windows machine, and save
the output in the file indicated by {\tt add-cross}.  Once you have done this,
you should be able to cross compile SML applications.  For example
\begin{verbatim}
mlton -host i386-pc-cygwin hello-world.sml
\end{verbatim}
will create {\tt hello-world.exe}, which you should be able to run from a Cygwin
shell on your Windows machine.

