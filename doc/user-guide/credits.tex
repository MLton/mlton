\sec{Credits}{credits}

\begin{itemize}

\item
{\mlton} was designed by:\\
\hspace*{2em}Henry Cejtin (\mailto{henry@\addr})\\
\hspace*{2em}Matthew Fluet (\mailto{fluet@cs.cornell.edu})\\
\hspace*{2em}Suresh Jagannathan (\mailto{sjagannathan@storagenetworks.com})\\
\hspace*{2em}Stephen Weeks (\mailto{sweeks@acm.org})

\item
Stephen Weeks implemented most of the original version of {\mlton}, and
continues to keep his fingers in most every part.

\item
Henry Cejtin wrote the {\tt IntInf} implementation,
the original profiling code,
the original man pages,
the {\tt .spec} files for the RPMs,
and lots of little hacks to speed stuff up.

\item
Matthew Fluet implemented the {\tt x86} native code generator,
ported {\tt mlprof} to work with the native code generator,
and did a lot of work on the SSA optimizer, both adding new optimizations
and improving or porting existing optimizations.

\item
Suresh Jagannathan implemented alternate inlining and uncurrying optimizations.

\item
\'{A}NOQ of the Sun (\mailto{anoq@HardcoreProcessing.com}) implemented {\tt
BinIO}, modified MLton so it could cross compile to MinGW, and provided useful
discussion about cross-compilation.

\item
Alain Deutsch (\mailto{deutsch@polyspace.com}) and \htmladdnormallink{PolySpace
Technologies}{http://www.polyspace.com/} provided many bug fixes and
runtime system improvements.

\item
Tom Murphy (\mailto{twm@andrew.cmu.edu}) wrote the original version of {\tt
MLton.Syslog} as part of his {\tt mlftpd} project and provided the description
of how to type check code using {\tt \_ffi} with SML/NJ.

\item
Michael Neumann (\mailto{uu9r@rz.uni-karlsruhe.de}) patched the runtime to
compile under FreeBSD 4.5.

\item
Barak Pearlmutter (\mailto{bap@cs.unm.edu}) maintains the
\htmladdnormallink{Debian packages}
		  {http://packages.debian.org/cgi-bin/search_packages.pl?keywords=mlton&searchon=names&subword=1&version=all&release=all}
for {\mlton}.

\item
{\mlton} was developed using
\htmladdnormallink{Standard ML of New Jersey}
		  {http://cm.bell-labs.com/cm/cs/what/smlnj/index.html}
and the
\htmladdnormallink{Compilation Manager (CM)}
		  {http://cm.bell-labs.com/cm/cs/what/smlnj/doc/CM/index.html}.

\item
{\mlton}'s lexer ({\tt src/frontend/ml.lex}), 
parser ({\tt src/frontend/ml.grm}),
and precedence-parser ({\tt src/elaborate/precedence-parse.fun})
are modified versions of code from the {\smlnj}.

\item
{\mlton} uses the {\smlnj} library implementation of splay trees.

\item
The {\mlton} basis library implementation uses modified versions of
portions of the the {\smlnj} basis library modules {\tt Real}, {\tt
IO}, {\tt Unix}, {\tt Process}, and {\tt Option}.

\item
The {\mlton} basis library implementation uses modified versions of
portions of the
\htmladdnormallink{ML Kit Version 3}
		  {http://www.it.edu/research/mlkit/kit3/readme.html}
basis library modules {\tt Real}, {\tt Path}, {\tt Time}, and
{\tt Date}.

\item
Many of the benchmarks come from the SML/NJ benchmark suite.

\item
Many of the regression tests come from the ML Kit Version 3 distribution, which
borrowed them from the
\htmladdnormallink{Moscow ML}
		  {http://www.dina.kvl.dk/~sestoft/mosml.html}
distribution.

\item
{\mlton} uses the
\htmladdnormallink{GNU multiprecision library}
		  {http://www.gnu.org/software/gmp/gmp.html}
for its implementation of {\tt IntInf}.

\item
We would like to thank to the following people for helpful discussions.\\
\hspace*{2em}Alain Deutsch (\mailto{deutsch@polyspace.com})\\
\hspace*{2em}Simon Helsen (\mailto{shelsen@acm.org})\\
\hspace*{2em}Richard Kelsey (\mailto{kelsey@research.nj.nec.com})\\
\hspace*{2em}Jeffrey Mark Siskind (\mailto{qobi@purdue.edu})\\
\hspace*{2em}\'{A}NOQ of the Sun (\mailto{anoq@HardcoreProcessing.com})
\end{itemize}
